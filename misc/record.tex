%%% Dependencies:
%%% ./ifempty.tex
%%% ./commandutils.tex

%%% This module adds very lightweight "record" types to LaTeX.
%%% The idea is that we can have objects like in Python or JavaScript:
%%% A data blob with a set of fields which we can get and set.
%%% (As of now, setting is constant: When we add a field to an object
%%% that field can neither be removed nor changed. This makes debugging
%%% much easier and matches most of my use cases.)
\makeatletter

%%% Conventions for implementing records.
%%% A record instance has an identifier, which is only used internally (e.g., vatras or phdulm).
%%% So these are basically variable names.
%%% Each record instance definition must produce macros named "<identifier>@<type><property name>".
%%% So for example, for my project "vatras", which is a "project" I can access its github stars by
%%%   "vatras@projectGithubStars"
%%% So a type definition of a record (such as those below) should create a set of macros of this naming scheme.
%%% Record names and field names are always in CamelCase starting capitalized (e.g., Job, JobBegin).

%%% Accesses a given field from a given record instance of a given type.
%%% This macro crashes if the given property was not defined!
%%% To check for that, use \hasprop.
%%% #1: instance name
%%% #2: type name
%%% #3: field name
\newcommand{\getfield}[3]{\runCommandByName{#1@#2@#3}}

%%% Sets the given field of a given record instance of a given type to a given value.
%%% Fields can only be set once!
%%% (We could allow overwriting existing values in the future but that is against the idea of a central, declarative data specification.)
%%% #1: instance name
%%% #2: type name
%%% #3: field name
%%% #4: desired value
\newcommand{\deffield}[4]{\defineCommandByName{#1@#2@#3}{#4}}

%%% conditionally checks whether a record instance has a field.
%%% KNOWN BUG:
%%% #1: record identifier
%%% #2: record type
%%% #2: attribute name; is always CamelCase (starting capitalized)
%%% #4: then branch
%%% #5: else branch
\newcommand{\ifhasfield}[5]{%
  %%% if the property command is defined ...
  \expandafter\ifdefined\csname #1@#2@#3\endcsname%
    %%% ... and not an empty value
    \ifelseempty{\getfield{#1}{#2}{#3}}%
      %%% ... else branch
      {#5}%
      %%% ... then branch
      {#4}%
  \else%
    %%% if the command is undefined, also run else branch
    #5%
  \fi%
}

%%% example definition of a record
%%%   \newcommand{\defJob}[6]{%
%%%     \deffield{#1}{Job}{Begin}{#2}%
%%%     \deffield{#1}{Job}{End}{#3}%
%%%     \deffield{#1}{Job}{Title}{#4}%
%%%     \deffield{#1}{Job}{Employer}{#5}%
%%%     \deffield{#1}{Job}{Description}{#6}%
%%%   }

\makeatother
